Tentamen består av fyra delar:
\begin{itemize}
\item
  Del A: Teorifrågor ($4 \times 2\mathrm{p} = 8\mathrm{p}$)  
\item
  Del B: Räkneuppgifter ($6 \times 3\mathrm{p} = 18\mathrm{p}$)
\item
  Del C: Programmering ($3 \times 3\mathrm{p} = 9\mathrm{p}$)
\item
  Del D: Problem ($3 \times 5\mathrm{p} = 15\mathrm{p}$)
\end{itemize}

\bigskip
\hrule
\bigskip

På del \textbf{A}, \textbf{B} och \textbf{C} skall \textbf{endast svar} anges.
Poäng ges endast för rätt svar; delpoäng ges endast i undantagsfall. 

På del \textbf{A} skall svaret anges som ett eller två ord.

På del \textbf{B} och del \textbf{C} skall svar anges exakt, dvs utan
avrundning. Uppgifterna är konstruerade så att det krävs maximalt tre decimaler.

På del \textbf{D} skall \textbf{fullständiga och välskrivna lösningar} lämnas
in. Dessa uppgifter kan också ge delpoäng för delvis fullständiga lösningar.

\bigskip
\hrule
\bigskip

Tentamen kan ge maximalt \textbf{50p}. Till detta läggs de bonuspoäng som
tjänats ihop under kursens gång. Betygsgränser är \textbf{20p (betyg 3)},
\textbf{30p (betyg 4)} och \textbf{40p (betyg 5)} för det sammanlagda
resultatet.

\bigskip

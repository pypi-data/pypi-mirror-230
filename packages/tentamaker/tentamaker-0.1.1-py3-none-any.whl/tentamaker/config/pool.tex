% A.1.1
Vad kallas en mängd $X$ som uppfyller
$\Forall x \in X \; \Exists \delta > 0 \Prop{B_{\delta}(x) \subseteq  X}$?
% Öppen
%

% A.2.1
Vad kallas en mängd $X$ som uppfyller
$\Forall x \in X \; \Exists \delta > 0 \Prop{B_{\delta}(x) \subseteq  X}$?
% Öppen
%

% A.3.1
Vad kallas en mängd $X$ som uppfyller
$\Forall x \in X \; \Exists \delta > 0 \Prop{B_{\delta}(x) \subseteq  X}$?
% Öppen
%

% A.4.1
Vad kallas en mängd $X$ som uppfyller
$\Forall x \in X \; \Exists \delta > 0 \Prop{B_{\delta}(x) \subseteq  X}$?
% Öppen
%

% B.1.1
Bestäm dubbelintegralen av funktionen $f(x, y) = 64(x^2 + y^2)/\pi$ över området
som begränsas av tredje kvadranten av enhets\-cirkeln och $r > \tfrac12$.
% $\frac{15}{2} = \num{7.5}$
Polära koordinater ger
\begin{align*}
  \int_{\pi}^{3\pi/2}\int_{1/2}^1 \frac{64r^2}{\pi} r\dr\dphi
  = \frac{64}{\pi}\frac{\pi}{2} \int_{1/2}^1 r^3\dr
  = 32 \cdot \frac{1}{4} \cdot \left(1 - \frac1{16}\right) = \frac{15}2
\end{align*}
%

% B.2.1
Bestäm dubbelintegralen av funktionen $f(x, y) = 64(x^2 + y^2)/\pi$ över området
som begränsas av tredje kvadranten av enhets\-cirkeln och $r > \tfrac12$.
% $\frac{15}{2} = \num{7.5}$
Polära koordinater ger
\begin{align*}
  \int_{\pi}^{3\pi/2}\int_{1/2}^1 \frac{64r^2}{\pi} r\dr\dphi
  = \frac{64}{\pi}\frac{\pi}{2} \int_{1/2}^1 r^3\dr
  = 32 \cdot \frac{1}{4} \cdot \left(1 - \frac1{16}\right) = \frac{15}2
\end{align*}
%

% B.3.1
Bestäm dubbelintegralen av funktionen $f(x, y) = 64(x^2 + y^2)/\pi$ över området
som begränsas av tredje kvadranten av enhets\-cirkeln och $r > \tfrac12$.
% $\frac{15}{2} = \num{7.5}$
Polära koordinater ger
\begin{align*}
  \int_{\pi}^{3\pi/2}\int_{1/2}^1 \frac{64r^2}{\pi} r\dr\dphi
  = \frac{64}{\pi}\frac{\pi}{2} \int_{1/2}^1 r^3\dr
  = 32 \cdot \frac{1}{4} \cdot \left(1 - \frac1{16}\right) = \frac{15}2
\end{align*}
%

% B.4.1
Bestäm dubbelintegralen av funktionen $f(x, y) = 64(x^2 + y^2)/\pi$ över området
som begränsas av tredje kvadranten av enhets\-cirkeln och $r > \tfrac12$.
% $\frac{15}{2} = \num{7.5}$
Polära koordinater ger
\begin{align*}
  \int_{\pi}^{3\pi/2}\int_{1/2}^1 \frac{64r^2}{\pi} r\dr\dphi
  = \frac{64}{\pi}\frac{\pi}{2} \int_{1/2}^1 r^3\dr
  = 32 \cdot \frac{1}{4} \cdot \left(1 - \frac1{16}\right) = \frac{15}2
\end{align*}
%

% B.5.1
Bestäm dubbelintegralen av funktionen $f(x, y) = 64(x^2 + y^2)/\pi$ över området
som begränsas av tredje kvadranten av enhets\-cirkeln och $r > \tfrac12$.
% $\frac{15}{2} = \num{7.5}$
Polära koordinater ger
\begin{align*}
  \int_{\pi}^{3\pi/2}\int_{1/2}^1 \frac{64r^2}{\pi} r\dr\dphi
  = \frac{64}{\pi}\frac{\pi}{2} \int_{1/2}^1 r^3\dr
  = 32 \cdot \frac{1}{4} \cdot \left(1 - \frac1{16}\right) = \frac{15}2
\end{align*}
%

% B.6.1
Bestäm dubbelintegralen av funktionen $f(x, y) = 64(x^2 + y^2)/\pi$ över området
som begränsas av tredje kvadranten av enhets\-cirkeln och $r > \tfrac12$.
% $\frac{15}{2} = \num{7.5}$
Polära koordinater ger
\begin{align*}
  \int_{\pi}^{3\pi/2}\int_{1/2}^1 \frac{64r^2}{\pi} r\dr\dphi
  = \frac{64}{\pi}\frac{\pi}{2} \int_{1/2}^1 r^3\dr
  = 32 \cdot \frac{1}{4} \cdot \left(1 - \frac1{16}\right) = \frac{15}2
\end{align*}
%

% C.1.1
Vilket tal är det tänkt att programmet skall räkna ut (variabeln \texttt{y})? \\
\begin{python}
from numpy.linalg import *

x = 1
y = 1

for k in range(1000):
    A = [[2*x, 2*y], [-2*x, 1]]
    b = [x**2 + y**2 - 42, y - x**2]
    sol = solve(A, b)
    x = x - sol[0]
    y = y - sol[1]

print(y)
\end{python}
% $6$
Programmet använder Newtons metod för att beräkna en lösning till ekvationssystemet
\begin{align*}
  x^2 + y^2 &= 42 \\
  y &= x^2
\end{align*}
Lösningarna är $(x, y) = (\pm\sqrt{6}, 6)$ och således $y = 6$.
%

% C.2.1
Vilket tal är det tänkt att programmet skall räkna ut (variabeln \texttt{y})? \\
\begin{python}
from numpy.linalg import *

x = 1
y = 1

for k in range(1000):
    A = [[2*x, 2*y], [-2*x, 1]]
    b = [x**2 + y**2 - 42, y - x**2]
    sol = solve(A, b)
    x = x - sol[0]
    y = y - sol[1]

print(y)
\end{python}
% $6$
Programmet använder Newtons metod för att beräkna en lösning till ekvationssystemet
\begin{align*}
  x^2 + y^2 &= 42 \\
  y &= x^2
\end{align*}
Lösningarna är $(x, y) = (\pm\sqrt{6}, 6)$ och således $y = 6$.
%

% C.3.1
Vilket tal är det tänkt att programmet skall räkna ut (variabeln \texttt{y})? \\
\begin{python}
from numpy.linalg import *

x = 1
y = 1

for k in range(1000):
    A = [[2*x, 2*y], [-2*x, 1]]
    b = [x**2 + y**2 - 42, y - x**2]
    sol = solve(A, b)
    x = x - sol[0]
    y = y - sol[1]

print(y)
\end{python}
% $6$
Programmet använder Newtons metod för att beräkna en lösning till ekvationssystemet
\begin{align*}
  x^2 + y^2 &= 42 \\
  y &= x^2
\end{align*}
Lösningarna är $(x, y) = (\pm\sqrt{6}, 6)$ och således $y = 6$.
%

% D.1.1
Bestäm flödet av vektorfältet $\bsF(x, y, z) = \frac{1}{\pi} (xy, yz, z^3)$
genom enhetssfären.
% $\frac{4}{5} = 0.8$
Gauss sats ger
\begin{align*}
  Q = \iint_{\partial\Omega} \bsF \cdot \Nhat \dS
  &= \iiint_{\Omega} \nabla \cdot \bsF \dV
    = \frac{1}{\pi} \iiint_{\Omega} y + z + 3z^2 \dV
\end{align*}
Integralen av de första två termerna blir noll pga symmetri. \\
Sfäriska koordinater och symmetri för den tredje termen ger
\begin{align*}
  Q &= \frac{1}{\pi} \iiint_{\Omega} x^2 + y^2 + z^2 \dV \\
    &= \frac{1}{\pi} \int_0^1 \int_0^{\pi} \int_0^{2\pi} r^2 \cdot
      r^2\sin(\theta)\dr\dtheta\dphi \\
    &= \frac{1}{\pi} \int_0^1 r^4 \dr \int_0^{\pi} \sin(\theta)
      \dtheta \int_0^{2\pi} \dphi \\
    &= \frac{1}{\pi} \cdot \frac{1}{5} \cdot 2 \cdot 2\pi
     = \frac{4}{5}
\end{align*}
%

% D.2.1
Bestäm flödet av vektorfältet $\bsF(x, y, z) = \frac{1}{\pi} (xy, yz, z^3)$
genom enhetssfären.
% $\frac{4}{5} = 0.8$
Gauss sats ger
\begin{align*}
  Q = \iint_{\partial\Omega} \bsF \cdot \Nhat \dS
  &= \iiint_{\Omega} \nabla \cdot \bsF \dV
    = \frac{1}{\pi} \iiint_{\Omega} y + z + 3z^2 \dV
\end{align*}
Integralen av de första två termerna blir noll pga symmetri. \\
Sfäriska koordinater och symmetri för den tredje termen ger
\begin{align*}
  Q &= \frac{1}{\pi} \iiint_{\Omega} x^2 + y^2 + z^2 \dV \\
    &= \frac{1}{\pi} \int_0^1 \int_0^{\pi} \int_0^{2\pi} r^2 \cdot
      r^2\sin(\theta)\dr\dtheta\dphi \\
    &= \frac{1}{\pi} \int_0^1 r^4 \dr \int_0^{\pi} \sin(\theta)
      \dtheta \int_0^{2\pi} \dphi \\
    &= \frac{1}{\pi} \cdot \frac{1}{5} \cdot 2 \cdot 2\pi
     = \frac{4}{5}
\end{align*}
%

% D.3.1
Bestäm flödet av vektorfältet $\bsF(x, y, z) = \frac{1}{\pi} (xy, yz, z^3)$
genom enhetssfären.
% $\frac{4}{5} = 0.8$
Gauss sats ger
\begin{align*}
  Q = \iint_{\partial\Omega} \bsF \cdot \Nhat \dS
  &= \iiint_{\Omega} \nabla \cdot \bsF \dV
    = \frac{1}{\pi} \iiint_{\Omega} y + z + 3z^2 \dV
\end{align*}
Integralen av de första två termerna blir noll pga symmetri. \\
Sfäriska koordinater och symmetri för den tredje termen ger
\begin{align*}
  Q &= \frac{1}{\pi} \iiint_{\Omega} x^2 + y^2 + z^2 \dV \\
    &= \frac{1}{\pi} \int_0^1 \int_0^{\pi} \int_0^{2\pi} r^2 \cdot
      r^2\sin(\theta)\dr\dtheta\dphi \\
    &= \frac{1}{\pi} \int_0^1 r^4 \dr \int_0^{\pi} \sin(\theta)
      \dtheta \int_0^{2\pi} \dphi \\
    &= \frac{1}{\pi} \cdot \frac{1}{5} \cdot 2 \cdot 2\pi
     = \frac{4}{5}
\end{align*}
%